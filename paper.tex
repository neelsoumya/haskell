\documentclass[12pt]{article}
\usepackage{fancyvrb}
\DefineVerbatimEnvironment{code}{Verbatim}{fontsize=\small}
\DefineVerbatimEnvironment{example}{Verbatim}{fontsize=\small}
\renewcommand\rmdefault{ptm}

\fontsize{12}{15}
\selectfont

\usepackage{amsfonts,graphicx,epsf,amsmath,amsbsy}
\title{CS 558, Homework 2}
\author{Soumya Banerjee}

\topmargin=-0.45in      %
\evensidemargin=0in     %
\oddsidemargin=-.2in      %
\textwidth=7in        %
\textheight=9.0in       %
\headsep=0.25in         %

\begin{document}
\maketitle

\clearpage
\newpage 

\section{Binary Trees}
\input{hw21.lhs}
\section{General Trees}
\input{hw22.lhs}
\section{Graphs}
\input{hw23.lhs}
\section{Numbers}
\input{hw24.lhs}

\section{HW 2.3 (Graphs)}
\subsection{1. Are all directed graphs representable in this fashion?}

No, all directed graphs cannot be represented in this fashion. The edges are represented in the form [(Int,Int)] and one would run out of numbers to enumerate the vertices if the number of vertices exceeded $2^{31} - 1$.

\section{HW 2.6 (Programs)}


foldr f e xs = foldl (flip f) e (reverse xs)

axioms:


foldr f v [] = v					 (Eq. 1)

foldr f v (x:xs) = f x (foldr f v xs)		 (Eq. 2)

foldl f v [] = v					 (Eq. 3)

foldl f v (x:xs) = foldl f (f v x) xs		 (Eq. 4)


flip f x y = f y x					 (Eq. 5)

reverse [] = []					 (Eq. 6)

reverse (x:xs) = (reverse xs) ++ [x]	 (Eq. 7)

[] ++ ys = ys						(Eq. 8)

(x:xs) ++ ys = x:(xs ++ ys)			(Eq. 9)

Proof by calculation


Let xs = [$x_{0}$, $x_{1}$, $x_{2}$, ......., $x_{n}$]       (Eq. 10)

LHS = foldr f e xs

	= \{ by Eq. 10 \}

	foldr f e [$x_{0}$, $x_{1}$, $x_{2}$, ......., $x_{n}$] 	
	
	= \{ by Eq. 2 definition of foldr \}
	
	f $x_{0}$ (foldr f v [$x_{1}$, $x_{2}$, ......., $x_{n}$] ) 
	
	= \{ by Eq. 2 definition of foldr \}
	
	f $x_{0}$ ( f $x_{1}$ (foldr f v [$x_{2}$,  ......., $x_{n}$] )) 

	= \{ by Eq. 2 definition of foldr \}
	
	f $x_{0}$ ( f $x_{1}$ (f $x_{2}$ (foldr f v [$x_{3}$,  ......., $x_{n}$] ))) 
	
	= \{ after applying Eq. 2 $n - 2$ times\}
	
	f $x_{0}$ ( f $x_{1}$ (f $x_{2}$ (.... (f $x_{n}$ (foldr f v [] ))... ))) 

	= \{ by Eq. 1 base case definition of foldr\}
	
	f $x_{0}$ ( f $x_{1}$ (f $x_{2}$ (.... (f $x_{n}$ e)... ))) 
	
	
	

RHS = foldl (flip f) e (reverse xs)

	 = \{ by Eq. 10 \}
	 
	 foldl (flip f) e (reverse [$x_{0}$, $x_{1}$, $x_{2}$, ......., $x_{n}$])	

	 = \{ by Eq. 7 definition of reverse \}
	 
	 foldl (flip f) e ((reverse [$x_{1}$, $x_{2}$, ......., $x_{n}$]) ++ [$x_{0}$] )
	 
	 = \{ by Eq. 7 definition of reverse \}
	 
	 foldl (flip f) e ( ((reverse [$x_{2}$, ......., $x_{n}$]) ++ [$x_{1}$]) ++ [$x_{0}$] )
	 
	 = \{ by Eq. 7 definition of reverse \}
	 
	 foldl (flip f) e ( ( ((reverse [$x_{3}$, ......., $x_{n}$]) ++ [$x_{2}$]) ++ [$x_{1}$]) ++ [$x_{0}$] )
	
	 = \{ after applying Eq. 7 $n - 2$ times \}
	 
	 foldl (flip f) e ( ( ( ( ...... ((reverse [ ]) ++ [$x_{n}$]) ++  .... ) ++ [$x_{2}$]) ++ [$x_{1}$]) ++ [$x_{0}$] )

	 = \{ after applying Eq. 6 base case of reverse \}
	 
	 foldl (flip f) e ( ( ( ( ...... ([ ] ++ [$x_{n}$]) ++  .... ) ++ [$x_{2}$]) ++ [$x_{1}$]) ++ [$x_{0}$] )

	 = \{ after applying Eq. 8 base case of ++ \}
	 
	 foldl (flip f) e ( ( ( ( ...... ([$x_{n}$] ++ [$x_{n-1}$]) .... ) ++ [$x_{2}$]) ++ [$x_{1}$]) ++ [$x_{0}$] )

	 = \{ after applying Eq. 9 definition of ++ \}
	 
	 foldl (flip f) e ( ( ( ( ...... ([$x_{n}$, $x_{n-1}$]) .... ) ++ [$x_{2}$]) ++ [$x_{1}$]) ++ [$x_{0}$] )

	 = \{ after applying Eq. 9 $n - 1$ times \}
	 
	 foldl (flip f) e ([$x_{n}$, $x_{n-1}$, ....., $x_{2}$, $x_{1}$, $x_{0}$])
	 
	 = \{ after applying Eq. 4 definition of foldl \}
	 
	 foldl (flip f) (flip f e $x_{n}$) ([$x_{n-1}$, ....., $x_{2}$, $x_{1}$, $x_{0}$])
	 
	 = \{ after applying Eq. 5 definition of flip \}
	 
	 foldl (flip f) (f $x_{n}$ e) ([$x_{n-1}$, ....., $x_{2}$, $x_{1}$, $x_{0}$])
	 
	 = \{ after applying Eq. 4 definition of foldl \}
	 
	 foldl (flip f) (flip f (f $x_{n}$ e) $x_{n-1}$) ([$x_{n-2}$, ....., $x_{2}$, $x_{1}$, $x_{0}$])
	 
	 = \{ after applying Eq. 5 definition of flip \}
	 
	 foldl (flip f) (f $x_{n-1}$ (f $x_{n}$ e)) ([$x_{n-2}$, ....., $x_{2}$, $x_{1}$, $x_{0}$])
	 
	 = \{ after applying Eq. 5 definition of flip and Eq. 4 definition of foldl $n - 4$ times \}
	 
	 foldl (flip f) (f $x_{3}$ (.... (f $x_{n-1}$ (f $x_{n}$ e)) ....) ) ([$x_{2}$, $x_{1}$, $x_{0}$])

	 = \{ after applying Eq. 4 definition of foldl \}
	 
	 foldl (flip f) (flip f (f $x_{3}$ (.... (f $x_{n-1}$ (f $x_{n}$ e)) ....) ) $x_{2}$) ([$x_{1}$, $x_{0}$])

	 = \{ after applying Eq. 5 definition of flip \}
	 
	 foldl (flip f) (f $x_{2}$ (f $x_{3}$ (.... (f $x_{n-1}$ (f $x_{n}$ e)) ....) ) ) ([$x_{1}$, $x_{0}$])

	 = \{ after applying Eq. 4 definition of foldl \}
	 
	 foldl (flip f) (flip f (f $x_{2}$ (f $x_{3}$ (.... (f $x_{n-1}$ (f $x_{n}$ e)) ....) ) ) $x_{1}$) ([$x_{1}$, $x_{0}$])

	 = \{ after applying Eq. 5 definition of flip \}
	 
	 foldl (flip f) (f $x_{1}$ (f $x_{2}$ (f $x_{3}$ (.... (f $x_{n-1}$ (f $x_{n}$ e)) ....) ) ) ) ([$x_{0}$])

	 = \{ after applying Eq. 4 definition of foldl \}
	 
	 foldl (flip f) (flip f (f $x_{1}$ (f $x_{2}$ (f $x_{3}$ (.... (f $x_{n-1}$ (f $x_{n}$ e)) ....) ) ) ) $x_{0}$) ([])

	 = \{ after applying Eq. 5 definition of flip \}
	 
	 foldl (flip f) (f $x_{0}$ (f $x_{1}$ (f $x_{2}$ (f $x_{3}$ (.... (f $x_{n-1}$ (f $x_{n}$ e)) ....) ) ) ) ) ([])

	 = \{ after applying Eq. 3 base case of foldl \}
	 
	 f $x_{0}$ (f $x_{1}$ (f $x_{2}$ (f $x_{3}$ (.... (f $x_{n-1}$ (f $x_{n}$ e)) ....) ) ) )
	 
	 
	
	
	Hence LHS = RHS by the calculational proof method and foldr f e xs = foldl (flip f) e (reverse xs).

	


\end{document}

%%%%%%%%%%%%%%%%%%%%%%%%%%%%%%%%%%%%%%%%%%%%%%%%%%%%%%%%%%%%%

